\documentclass{bmvc2k}
\usepackage{multicol}
\usepackage{minted}
\usepackage{float}

%% Enter your paper number here for the review copy
% \bmvcreviewcopy{??}

\title{OpenMP -- podstawy}

% Enter the paper's authors in order
% \addauthor{Name}{email/homepage}{INSTITUTION_CODE}
\addauthor{Maciej A. Czyzewski}{inf136698}{1}

% Enter the institutions
% \addinstitution{Name\\Address}
\addinstitution{
	Poznan University of Technology \\
	Poland
}

\runninghead{Raport}{Programowanie Rownolegle}

% Any macro definitions you would like to include
% These are not defined in the style file, because they don't begin
% with \bmva, so they might conflict with the user's own macros.
% The \bmvaOneDot macro adds a full stop unless there is one in the
% text already.
\def\eg{\emph{e.g}\bmvaOneDot}
\def\Eg{\emph{E.g}\bmvaOneDot}
\def\etal{\emph{et al}\bmvaOneDot}

%-------------------------------------------------------------------------
% Document starts here
\begin{document}

\maketitle

%-------------------------------------------------------------------------

% --> [FINAL]
% - tabela zawierajaca czas obliczen dla P1-P7
% - wspolczynnik przyspieszenia dla ronwlelglo + ilosc watkow
% - omowienie dla kazdej wersji dlaczego wolniej/szybciej jest
% - okreslenie dlugosc lini wg. PI7
% - opisanie PI7 wlasnymi slowami

\section{Dane techniczne}

\begin{table}[H]
\begin{center}
\begin{tabular}{|l|c|}
\hline
Processor Name & Intel Core i5 \\
Processor Speed &	2.6 GHz \\
Number of Processors &	1 \\
Total Number of Cores &	2 \\
Logic Cores (max\_threads) &	4 \\
L2 Cache (per Core) &	256 KB \\
L3 Cache &	3 MB \\
Memory &	8 GB \\
Kernel Version &	Darwin 17.7.0 \\
Compiler & g++ 9.3.0 \\
\hline
\end{tabular}
\end{center}
\end{table}

\section{Rezultaty}

Niektore wyjasnienia/spostrzezenia znaduja sie w implementacjach.

\begin{table}[H]
\begin{center}
\begin{tabular}{|l|c|c|c|c|r|}
\hline
Task & {\tt LOGIC} & {\tt PHYSICAL} & {\tt HALF} & average &
ratio \\
\hline\hline
PI1 & 0.496406 & 0.496406 & 0.496406 & 0.496406 & 1 \\
PI2 & 0.720502 & 0.433100 & 0.518239 & 0.557280 & 0.89 \\
PI3 & 6.735107 & 4.300983 & 1.378666 & 4.138252 & 0.11 \\
PI4 & \textbf{0.240313} & 0.259092 & \textbf{0.481288} & \textbf{0.326897} &
\textbf{1.51} \\
PI5 & 0.244805 & \textbf{0.250948} & 0.488539 & 0.328097 & \textbf{1.51} \\
PI6 & 0.531926 & 0.282501 & 0.505163 & 0.439863 & 1.12 \\

\hline
\end{tabular}
\end{center}
\end{table}

\begin{itemize}
	\item \textbf{PI1}: nasz referencyjny kod sekwencyjny, ze wzgledu na to ze u
		mnie zmienna \\ {\tt THREADS\_HALF = 2 / 2 = 1}. Mozemy w trzeciej kolumnie
		porownac jak rozne implementacje rownolegle ``dodaja'' czas przetwarzania dla pojedynczego
		thread-u.
	\item \textbf{PI2}: wspoldzielenie wszystkich wartosci mocno spowalnia kod
		(zmienne globalne maja wolniejsze zapisy/odczyty), dodatkowo kod jest
		nie prawidlowy bo rozne watki uczestnicza w wyscigu (zly wynik $\Pi$).
	\item \textbf{PI3}: jest znaczaco wolniejsza poniewaz wieloktornie w petli
		wywolywana jest ``atomic'' ktory jest bardzo kosztowny.
	\item \textbf{PI4}: przesuniecie ``atomic''-a z petli pozwala wywolania go
		tylko tyle razy ile mamy thread-ow (a wiec znaczace przyspieszenie).
	\item \textbf{PI5}: ``reduction'' dziala podobnie jak ``PI4`` nie trzeba
		tylko przechowywac sum czesciowych wlasnej implementacji.
	\item \textbf{PI6}: tutaj mamy ``false sharing'' dlatego bedzie wolniej,
		jednak gdy sie znajdzie odpowiedni padding (uzywajac PI7), uzyskujemy
		takie same czasy jak w PI4/PI5.
\end{itemize}

\section{PI7: obliczanie rozmiaru ``cache line''}

Postanowilem ze zaimplementuje troche inny ale rownowazny eksperyment.
Program bedzie zmienial wartosc {\tt memshift} ktora odpowiada za padding.
Dodatkowo ustawiamy \\ {\tt schedule(static, 1) nowait} aby miec podzial pracy
statyczny cykliczny oraz tylko 2 watki. W ten sposob mamy pewnosc ze beda na
przemian pobierac swoje linie pamieci.

Napoczatku kiedy {\tt memshift=0} wszystkie wartosci sa obok siebie w wektorze
{\tt vsum}, a to oznacza ze watki beda pobieraly ten sam wiersz (false sharing).
Czas wykonania powinnien byc dluzszy niz w PI4. Teraz iterujac po coraz
wiekszych paddingach probojemy znalesc pierwsza ktora dala znaczacy spadek czasu
wykonania. To oznacza ze drugi watek zaczal korzystac z kolejnej linii.
Na bazie tej odleglosci mozemy obliczyc dlugosc linii.

Z eksperymentu wynika ze {\tt memshift=7} (bo czasy sa takie jak w PI4)
a wiec dlugosc linii wynosi 8 $\cdot$ 8 bajtow $=$ 64 na dane
(co jest zgodne z moja architektura komputera).

\begin{minted}[mathescape,
               linenos,
               numbersep=5pt,
               gobble=0,
               frame=lines,
			   framesep=2mm]{c++}
#define THREADS_POLICY 2
#define NUM_STEPS 1000000000
#define MEMSHIFT 16
volatile double vsum[THREADS_POLICY * (MEMSHIFT + 1)] = {0};
// . . .
// iterate for best memshift (minimize time)
// . . .
#pragma omp for schedule(static, 1) nowait
      for (int i = 0; i < NUM_STEPS; i++)
        vsum[idx + idx * memshift] +=
            4.0 / (1. + (i + .5) * (i + .5) * (step2));
    }
\end{minted}

\newpage

\subsection{Analiza {\tt memshift}}

\begin{multicols}{2}
\begin{minted}[mathescape,
               linenos,
               numbersep=5pt,
               gobble=0,
               frame=lines,
			   fontsize=\small,
			   framesep=2mm]{python}
threads_num=2

MEMSHIFT=0
  0 -> vsum[ 0]   1 -> vsum[ 1]
<time.h> time=8.232575
 <omp.h> time=4.170651

MEMSHIFT=1
  0 -> vsum[ 0]   1 -> vsum[ 2]
<time.h> time=7.778722
 <omp.h> time=3.921388

MEMSHIFT=2
  0 -> vsum[ 0]   1 -> vsum[ 3]
<time.h> time=7.801917
 <omp.h> time=3.932336

MEMSHIFT=3
  0 -> vsum[ 0]   1 -> vsum[ 4]
<time.h> time=7.810248
 <omp.h> time=3.949122

MEMSHIFT=4
  0 -> vsum[ 0]   1 -> vsum[ 5]
<time.h> time=8.097199
 <omp.h> time=4.086339

MEMSHIFT=5
  0 -> vsum[ 0]   1 -> vsum[ 6]
<time.h> time=7.827552
 <omp.h> time=3.954192

MEMSHIFT=6
  0 -> vsum[ 0]   1 -> vsum[ 7]
<time.h> time=7.848424
 <omp.h> time=3.952066









MEMSHIFT=7 (!)
  0 -> vsum[ 0]   1 -> vsum[ 8]
<time.h> time=5.211482
 <omp.h> time=2.619698

MEMSHIFT=8
  0 -> vsum[ 0]   1 -> vsum[ 9]
<time.h> time=5.438142
 <omp.h> time=2.752816

MEMSHIFT=9
  0 -> vsum[ 0]   1 -> vsum[10]
<time.h> time=5.214670
 <omp.h> time=2.624130

MEMSHIFT=10
  0 -> vsum[ 0]   1 -> vsum[11]
<time.h> time=5.413838
 <omp.h> time=2.740313

MEMSHIFT=11
  0 -> vsum[ 0]   1 -> vsum[12]
<time.h> time=5.261158
 <omp.h> time=2.648972

MEMSHIFT=12
  0 -> vsum[ 0]   1 -> vsum[13]
<time.h> time=5.504936
 <omp.h> time=2.802603

MEMSHIFT=13
  0 -> vsum[ 0]   1 -> vsum[14]
<time.h> time=5.456885
 <omp.h> time=2.780098

MEMSHIFT=14
  0 -> vsum[ 0]   1 -> vsum[15]
<time.h> time=5.604465
 <omp.h> time=2.878687

MEMSHIFT=15
  0 -> vsum[ 0]   1 -> vsum[16]
<time.h> time=5.297931
 <omp.h> time=2.668634
\end{minted}
\end{multicols}

\end{document}
